\documentclass[12pt,a4paper]{article}
\usepackage[utf8]{inputenc}
\usepackage{amsmath}

\usepackage{amsfonts}
\usepackage{amssymb}
\usepackage{tikz}
\usepackage{amsmath}
\usepackage{amssymb}
\usepackage{pgfplots}
\usepackage{nccmath}
\usepackage{mathtools}
\usepackage{pgfplots}
\usepackage{mathtools,amssymb}
\usepackage{tikz}
\usepackage{xcolor}
\pgfplotsset{compat = newest}
\author{Chris Camano: ccamano@sfsu.edu}
\title{MATH301 Homework 2 }
\date{2/8/2022}
% Margins
\topmargin=-0.45in
\evensidemargin=0in
\oddsidemargin=0in
\textwidth=6.5in
\textheight=9.0in
\headsep=0.25in
\newcommand{\q}{\quadd}
\renewcommand{\labelenumi}{\alph{enumi})}
\newcommand{\rtwo}{$\mathbb{R}^2$}
\newcommand{\C}{$\mathbb{C}$}
\newcommand{\tf}{\therefore}
\begin{document}
\maketitle
\section{Section 1.5 Problems:4a,4e,4f,4g,8}\\
$$ \text{Suppose} A=\{b,c,d\}\text{ and } B=\{a,b\} \text{ Find}:$$
\textbf{1.5:4a}\\
\[
  (A \times B) \cap ( B \times B)
\]
\begin{align*}
  (A \times B)=\{(b,a),(b,b),(c,a),(c,b),(d,a),(d,b)\}\\
  (B \times B)=\{(a,a),(a,b),(b,a),(b,b))\}\\
  \therefore \\
    (A \times B) \cap ( B \times B)=\{(b,a),(b,b)\}
\end{align*}\\\\
\textbf{1.5:4e}\\
\[
  (A \times B) \cap B
\]
\begin{align*}
  (A \times B)=\{(0,1),(0,2),(1,1),(1,2)\}\\
  1,2 \notin (A \times B)\quadd \therefore (A \times B) \cap B=\emptyset
\end{align*}
\textbf{1.5:4f}\\
\[
  \mathcal{P}(A)\cap \mathcal{P}(B)
\]
\begin{align*}
  \mathcal{P}(A)=\{\emptyset, \{b\},\{c\},\{d\},\{b,c\},\{b,d\},\{c,d\},\{b,c,d\}\\
  \mathcal{P}(B)=\{\emptyset,\{a\},\{b\},\{a,b\}\\ \therefore
    \mathcal{P}(A)\cap \mathcal{P}(B)=\{\emptyset,\{b\}
\end{align*}
\textbf{1.5:4g}\\
\[
    \mathcal{P}(A) - \mathcal{P}(B)
\]
\begin{align*}
  \mathcal{P}(A)=\{\emptyset, \{b\},\{c\},\{d\},\{b,c\},\{b,d\},\{c,d\},\{b,c,d\}\\
  \mathcal{P}(B)=\{\emptyset,\{a\},\{b\},\{a,b\}\\\tf
  \mathcal{P}(A)-\mathcal{P}(B)=\{x|x\in\mathcal{P}(A),x\notin\mathcal{P}(B)\}=\{\{c\},\{d\},\{b,c\},\{b,d\},\{c,d\},\{b,c,d\}\\
\end{align*}
\textbf{1.5:8}\\
See attached illustration, figure 1

\section{Section 1.7 Problems 6,8,12,14}\\
\textbf{1.7.6}\\
See attached illustration, figure 2\\
\textbf{1.7.8}\\
See attached illustration, figure 3\\
\textbf{1.7.12}\\
The expression that describes this set it:
\[
(A-B)\cup (B\cap C)
\]
\textbf{1.7.14}\\
The expression that describes this set is :
\[
  (A\cap B\cap C)\cap \Big((A-C)\cap(A-B)\Big)
\]
\section{4 dimensional cube illustration}\\
See attached illustration, figure 4
\section{Bonus }
\end{document}
