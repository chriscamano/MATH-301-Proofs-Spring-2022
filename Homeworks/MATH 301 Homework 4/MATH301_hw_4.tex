\documentclass[12pt,a4paper]{article}
\usepackage[utf8]{inputenc}
\usepackage{amsmath}
\usepackage{enumitem}
\usepackage{amsfonts}
\usepackage{amssymb}
\usepackage{tikz}
\usepackage{amsmath}
\usepackage{amssymb}
\usepackage{pgfplots}
\usepackage{nccmath}
\usepackage{mathtools}
\usepackage{pgfplots}
\usepackage{mathtools,amssymb}
\usepackage{tikz}
\usepackage{xcolor}
\pgfplotsset{compat = newest}
\author{Chris Camano: ccamano@sfsu.edu}
\title{MATH 301 Homework 4 }
\date{2/17/2022}
% Margins
\topmargin=-0.45in
\evensidemargin=0in
\oddsidemargin=0in
\textwidth=6.5in
\textheight=9.0in
\headsep=0.25in
\newcommand{\q}{\quad}
\renewcommand{\labelenumi}{\alph{enumi})}
\newcommand{\R}{\mathbb{R}}
\newcommand{\rtwo}{$\mathbb{R}^2$}
\newcommand{\C}{$\mathbb{C}$}
\newcommand{\sect}[1]{\section*{#1}}
\begin{document}
\maketitle
%\[
%  p_{ij}=\frac{\frac{e^{-||x_i-x_j||_2^2/2\sigma_i^2}}{\sum_{k\neq i}e^{-||x_i-x_k||_2^2/2\sigma_i^2}}+\frac{e^{-||x_j-x_i||_2^2/2\sigma_j^2}}{\sum_{k\neq j}e^{-||x_j-x_k||_2^2/2\sigma_j^2}}}{2N}
%\]
%\[
%  p_{ij}=\frac{\Big[\sum_{k\neq j}e^{\frac{-||x_i-x_j||_2^2-||x_j-x_k||_2^2}{4\sigma_i^2\sigma_j^2}}
%  +
%  \sum_{k\neq i}e^{\frac{-||x_j-x_i||_2^2-||x_i-x_k||_2^2}{4\sigma_i^2\sigma_j^2}}\Big]2N}{\sum_{k \neq i}\sum_{l\neq j}e^{\frac{-\sigma_j^2||x_j-x_k||_2^2-\sigma_i^2||x_j-x_l||_2^2}{2\sigma_i^2\sigma_j^2}}}
%\]%
%%%%%%%%%%%%%%%%%%%%%%%%%%%%%%%%%%%%%%%%%%%%%%%%%
\sect{Problem 1.Negate the following implications.}
\begin{proof}
  \begin{itemize}[label={--}]
    \item \textbf{(a)+++}\\ If I finish my homework early, I will join you for a movie.\\
    Let I finish my homework be P and I will join you for a movie be Q $\therfore$
    \[
      P \rightarrow Q
    \]
    The negation:
    \[
      \neg (P \rightarrow Q)= P \land \neg Q
    \]
    This negation makes sense to me and can be thought of as a lie! The negation encapsulates the concept of finishing homework early and not going to the movie.
    \item  \textbf{(b)+++}\\ If $n^2$ is divisible by 12, then n is even or n is a multiple of three.\\
    An interpretation of this statement learned from the in class activity is:
    \[
      \forall n \in \mathbb{Z} : 12|n^2, (n \in \mathbb{E}) \lor (3|n)
    \]
    The negation:
    \[
      \neg(   \forall n \in \mathbb{Z} : 12|n^2, (n \in \mathbb{E}) \lor (3|n))=
    \]
    \[
      \exists n \in \mathbb{Z} : 12|n^2 \land ( n \notin \mathbb{E}) \land (3\nmid n))
    \]
    Which can be understood as the statement there exists some integer n such that 12 divides $n^2$ and n is not even and 3 does not divide n. After our disussion in class about this problem it makes sense to me and is a good exercise on identifying situations where we are making universal claims instead of singular ones.
  \end{itemize}
\end{proof}
%%%%%%%%%%%%%%%%%%%%%%%%%%%%%%%%%%%%%%%%%%%%%%%%%
\sect{Problem  2. Consider the two statements: \textbf{+++}}
\newcommand{\Z}{\mathbb{Z}}
\begin{proof}
  P: For all integers x, there exists an integer y such that $y > x.$
  Q: There exists an integer x such that, for all integers y, we have $y > x.$
  \\
  Let us first start by translating these statements to their  logical forms:
  \[
    P=\forall x \in \Z(\exists y \in \Z: y>x)
  \]
  \[
    Q=(\exists x \in \Z: \forall y \in \Z,y>x)
  \]
  \begin{itemize}[label={--}]
    \item   (a)  One of these statements is true and the other is false. Determine which statement is true and justify your claim.\\\\
    The first statement P is true. For all integers x you can always find some integer y such that the value of y is greater than the value of x. This is due to the infinite nature of the integers, meaning you can always look ahead past your selection of x to find a greater value.
    \item   (b)  Discuss why the other statement is false.\\\\
    Q is false because  $\nexists x: \forall y \in \Z, y>x$ This is to say that there is no single value x that is smaller than all other integers. Because the intergers are infinite whatever value we pick for x there will always exist some integer y with a lower value.
    \item  (c)  Write down the negation of each statements.
    \[
      \neg P=\neg (\forall x \in \Z(\exists y \in \Z: y>x)=\exists x \in \Z(\forall y \in \Z: y \leq x)
    \]
    \[
    \neg Q= \neg(\exists x \in \Z: \forall y \in \Z,y>x)=\forall x \in \Z: \exists y \in \Z, y \leq x
    \]
  \end{itemize}
  Overall this question was very fun and good practice with nested quantifiers. I think that I understand how to translate to the negated form the trickiest part for me is placing the such that at the end of the negation.
\end{proof}
%%%%%%%%%%%%%%%%%%%%%%%%%%%%%%%%%%%%%%%%%%%%%%%%%
\sect{Problem 3. \textbf{+++} Consider the statement:}
\begin{proof}
  “For every positive integer n, there exists a prime number between n and n + 10.”\\
  \begin{itemize}[label={--}]
    \item (a) Do you think the statement is true or false. Why?\\\\
    I think that this statement is false due to a very compelling counterargument identified by my in class partner Ian which states that if n is equal to 200 there is not a prime number between 200 and 210.This can be verified computationally with most programming languages
    \item (b) Write down the negation of this statement.
    \\\\
    The statement is first stated as follows:
    \[
      \forall n \in \Z^+(\exists p \in \textbf{P}: n<p<n+10)
    \]
    where \textbf{P} is the set of prime numbers. \\\\
    The Negation:
    \[
      \neg (\forall n \in \Z^+(\exists p \in \textbf{P}: n<p<n+10))=\exists n \in \Z^+(\forall p \in \textbf{P}:(n\geq p)\land (n+10 \geq p))
    \]
    This negation reads: there exists some positive integer n forall primes such that n is greater than or equal to a given prime p and n+10 is also greater than or equal to that same p
  \end{itemize}
\end{proof}
%%%%%%%%%%%%%%%%%%%%%%%%%%%%%%%%%%%%%%%%%%%%%%%%%
\sect{Problem 4}
\begin{proof}
  \begin{itemize}[label={--}]
    \item (a) 2.10.2 \textbf{+++} \\\\
  If x is prime then $\sqrt{x}$ is not a rational number
  Let P be "x is prime"\\
  Let Q  be "$\sqrt{x}$ is not rational"\\
  \[
    P \rightarrow Q
  \]
  \[
    \neg (P \rightarrow Q)= P \land \neg Q, \therefore
  \]
  The negation is: "x is prime and $\sqrt{x}$ is rational"\\
  I am confident with this problem and have built up an understanding that negating statement is a way of describing the opposite.
  \item (b) 2.10.4 \textbf{+++}\\\\
  For every positive number $\epsilon$, there is a positive number $\delta$ such that $|x-a|<\delta$ implies $|f(x)-f(a)|<\epsilon$
  \[
    \forall \epsilon \in \R^+(\exists \delta \in \R^+:(|x-a|<\delta) \rightarrow (|f(x)-f(a)|<\epsilon) )
  \]
  The negation:
  \[
    \exists \epsilon \in \R^+:\forall \delta\in \R^+, (|x-a|<\delta)\land (|f(x)-f(a)|\geq\epsilon))
  \]
  The negation of this statement reads there exists some positive real number $\epsilon$ (Here number is assumed to be real technically we could just pick some element of a field I suppose) such that for all real numbers $\delta,  |x-a|<\delta$ and $(|f(x)-f(a)|\geq\epsilon$
  \end{itemize}
\end{proof}
%%%%%%%%%%%%%%%%%%%%%%%%%%%%%%%%%%%%%%%%%%%%%%%%%
\sect{Problem 5}
\begin{proof}
  \begin{itemize}[label={--}]
    \item(a) 2.10.8\textbf{+++}\\\\
    If x is a rational number and x $\neq$ 0, then tan(x) is not a rational number:
    \[
      (x\in \mathbb{Q}\land x\neq 0)\rightarrow (tan(x)\notin \mathbb(Q))
    \]
    The negation:
    \[
      (x\in \mathbb{Q}\land x\neq 0) \land (tan(x) \in \mathbb{Q})
    \]
    For this problem i deployed the negatio of the conditional statement to find my answer. I am confident in this result!
    \item(b) 2.10.10 \textbf{+++}\\\\
    If f is a polynomial and its degree is greater than 2 then $f^'$ is not constatnt.
    \[
      (f \in \mathcal{P}_n\land n>2)\rightarrow f^{'}\text{ is not constant}
    \]
    The negation:
    \[
      (f \in \mathcal{P}_n\land n>2)\land f^{'}\text{ is  constant}
    \]
    This was a good problem that made me consider how to represent the set of polynomial over a certain degree. I chose the notation above from referencing a math stack overflow post. I find it to be an intuitive way of describing the set as there is a handy subscript at all times that denotes the degree of the polynomial family being sampled.
  \end{itemize}


\end{proof}
\end{document}
%%%%%%%%%%%%%%%%%%%%%%%%%%%%%%%%%%%%%%%%%%%%%%%%%
