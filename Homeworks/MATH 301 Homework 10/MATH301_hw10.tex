\documentclass[12pt]{article}
\usepackage[pdftex]{graphicx}
\usepackage{amsmath}
\usepackage{amssymb}
\pagestyle{empty}
\author{Chris Camano: ccamano@sfsu.edu}
\title{MATH 301  Homework 10 }
\date{4/25/2022}

\topmargin -0.6in
\headsep 0.40in
\oddsidemargin 0.0in
\textheight 9.0in
\textwidth 6.5in

\newcommand{\econst}{\mathrm{e}}
\newcommand{\diff}{\mathrm{d}}
\newcommand{\dwrt}[1]{\frac{\diff}{\diff #1}}
%%%%%%Macros for 425%%%%%%%%%%%%%%%%%%%
\newcommand{\q}{\quad}
\newcommand{\tab}{\\\\}
\renewcommand{\labelenumi}{\alph{enumi})}
\newcommand{\sect}[1]{\section*{#1}}

%%%%%%Vector Spaces%%%%%%%%%%%%%%%%%%%
\newcommand{\R}{\mathbb{R}}
\newcommand{\C}{\mathbb{C}}
\newcommand{\F}{\mathbb{F}}
\newcommand{\rtwo}{\mathbb{R}^2}
\newcommand{\mxn}{{mxn}}

%%%%%%Sets and common phrases%%%%%%%%%
\newcommand{\Axb}{\textbf{Ax=b} }
\newcommand{\Axz}{\textbf{Ax=0} }
\newcommand{\dim}{\text{dim}}
\newcommand{\lc}{linear combination }
\newcommand{\let}{\text{Let }}
\newcommand{\tf}{\therefore}
%%%%%%%%%%%%%%%%%%%%%%%%%%%%%%%%%%%%%%%
\everymath={\displaystyle}


\begin{document}
\maketitle
\sect{Problem 9.2}
\begin{proof}
  +++\\
  For every natural number n, the integer $2n^2-4n+31$ is prime\\
  This statement is false. Consider any of the following numbers\\
  These are the first examples up to 500 for which the statement above does not hold. \\\\
  30 ,31 ,33 ,36 ,40 ,45 ,51 ,58 ,59 ,62 ,64 ,66 ,73 ,75 ,77 ,85 ,88 ,89 ,90 ,92 ,93 ,95 ,96 ,98 ,100 ,103 ,108 ,110 ,114 ,117 ,119 ,121 ,123 ,124 ,126 ,127 ,128 ,135 ,139 ,145 ,146 ,147 ,148 ,149 ,150 ,151 ,154 ,155 ,157 ,158 ,164 ,165 ,166 ,167 ,170 ,175 ,176 ,180 ,181 ,183 ,184 ,186 ,188 ,192 ,195 ,197 ,201 ,204 ,205 ,206 ,207 ,208 ,209 ,210 ,211 ,214 ,217 ,219 ,220 ,221 ,225 ,233 ,234 ,237 ,238 ,239 ,240 ,241 ,243 ,248 ,249 ,250 ,258 ,261 ,262 ,263 ,264 ,265 ,266 ,267 ,272 ,275 ,279 ,280 ,281 ,283 ,286 ,289 ,291 ,292 ,294 ,295 ,298 ,299 ,301 ,302 ,303 ,304 ,305 ,306 ,307 ,310 ,312 ,320 ,321 ,322 ,323 ,325 ,326 ,327 ,330 ,332 ,333 ,335 ,336 ,340 ,341 ,343 ,344 ,348 ,349 ,352 ,353 ,354 ,355 ,359 ,363 ,368 ,369 ,371 ,372 ,373 ,374 ,375 ,377 ,378 ,379 ,380 ,381 ,384 ,385 ,389 ,390 ,391 ,392 ,393 ,396 ,399 ,401 ,403 ,405 ,406 ,407 ,408 ,409 ,410 ,411 ,414 ,418 ,421 ,422 ,424 ,427 ,432 ,433 ,434 ,435 ,436 ,437 ,438 ,439 ,441 ,443 ,444 ,445 ,447 ,449 ,450 ,452 ,454 ,455 ,456 ,460 ,464 ,465 ,467 ,468 ,469 ,470 ,474 ,478 ,479 ,480 ,482 ,483 ,484 ,485 ,486 ,488 ,491 ,492 ,493 ,494 ,495 ,496 ,498 ,499 ,500\\\\
  \\ all of these numbers are produced by the function above and are not prime. computed with c++.\\\\
  \textbf{Commentary:}\\
  I enjoyed this problem because I found that counter examples were readily available. This allowed me to use my computer to confirmmy suspicion
\end{proof}
\sect{Problem 9.6}
\begin{proof}
  +++\\
  If A,B,C,D are sets then :
  \[
    (A\times B)\cap (C\times D)=(A\cap C)\times(B\cap D)
  \]
  \textbf{Part 1 }
  \[
    (A\times B)\cap (C\times D)\subseteq(A\cap C)\times(B\cap D)
  \]
  Let (x,y) $\in (A\times B)\cap (C\times D)$. Then (x,y) is an element of $(A\times B)\land (C\times D)$.This means that x is an element of A and y is an element of B and that x is also an element of C and that y is also an element of D. \\
  Since x is an element of A and C it is an element of $A\cap C$ \\
  Since y is an element of B and D it is an element of $B\cap D$\\
  This implies that $(x,y) \in (A\cap C)\times(B\cap D)$ since all ordered pairs x,y have the property that $x\in A\cap C$ and $y\in B\cap D$\\\\
  \textbf{Part 2 }
  \[
    (A\cap C)\times(B\cap D)\subseteq(A\times B)\cap (C\times D)
  \]
  let (x,y) be an element of $  (A\cap C)\times(B\cap D)$. Then x is an element of $A\cap C$ and y is an element of $B\cap D$ . Since $x \in A \land y \in B$ (x,y)$\in A \times B$. \\ Since $x \in C \land y \in D$ (x,y)$\in C \times D$. Thus (x,y) is an element of $\in A \times B$ and $\in C \times D$, so (x,y) $\in (A\times B)\cap (C\times D)$ which implies $(A\cap C)\times(B\cap D)\subseteq(A\times B)\cap (C\times D)$
  \\\\
  \textbf{Commentary:}\\
  This proof was easy for one direction but hard for the other. Proving that ordered pairs have certain properties was harder than proving properties for a single element.
\end{proof}
\sect{Problem 9.14}
\begin{proof}
  +++\\
  If A and B are sets then :
  \[
    \mathCal{P}(A)\cap\mathcal{P}(B)=\mathcal{P}(A\cap B)
  \]
  \begin{align*}
    &\mathCal{P}(A)\cap\mathcal{P}(B)\subseteq \mathcal{P}(A\cap B):\\
  \end{align*}
  Let x be an element of $\mathCal{P}(A)\cap\mathcal{P}(B)$ then $x\in \mathCal{P}(A) \land x \in \mathCal{P}(B)$ This implies that since x is in the power set of A and the power set of  B that it is a subset A and a subset of B. Since x is a subset of A and B
  \[
    X \in \mathcal{P}(A\cap B)
  \]
  \begin{align*}
    &\mathcal{P}(A\cap B)\subseteq \mathCal{P}(A)\cap\mathcal{P}(B):\\
  \end{align*}
  Let x be an element of $\mathcal{P}(A\cap B)$. Then x is a subset of $A\cap B$. since all subsets of the intersection of A and B are subsets of the powerset of A and the powerset of B respectively this implies.
  \[
    x\in \mathCal{P}(A)\cap\mathcal{P}(B)
  \]\\\\
  \textbf{Commentary:}\\
  I think that this proof was my favorite on the assignment because of how effective the definition of power sets was for making arguments here. I used to dislike powerset proofs quite a bit but they have developed a distinct tone that I really enjoy thinking about.
\end{proof}
\sect{Problem 9.18}
\begin{proof}
  +++\\
  If a,b,c$\in \mathbb{N}$ then at least one of a-b,a+c and b-c is even.
  \\
  \textbf{Proof by contradiction}. Assume that all three of the relationships above are in fact odd. Then the sum of the terms a-b and b-c must be odd since :
  \[
    (b-c)+(a-b)=a-c
  \]
  and a-c is one of the numbers declared odd. However the sum of two odd numbers must be even so we arrive at a contradiction  proving the original statement by contraposition.\\
  \textbf{Commentary}\\
  This proof was very difficult and started out as trying out many cases in order to examine the effect each case had on the outcome. After doing research online regarding proofs of three elements I found this technique of using contraposition to lead to contradiction by self referencing the relationships of the previously defined elements in order to illustrate a contradiction that defies the properties of even and off numbers.
\end{proof}
\sect{Problem 9.24}
\begin{proof}
  +++\\
  The inequality $2^x\geq x+1$ is true for all positive real numbers x.

  Consider the value x=$\frac{1}{3}$\\
  \begin{align*}
    &2^\frac{1}{3}\geq \frac{1}{3}+1\\
    &1.25992 \geq 1.3333
  \end{align*}
  and it is shown that the statment is no longer true. Thus by counter example we demonstrate the statement above is false.\\\\
  \textbf{Commentary:}\\
  This proof was very straightforward and I found a counter example to disprove it. I am not sure if all rational numbers less than one satisfy this property.
    \end{proof}
\sect{Bonus}
\end{document}
