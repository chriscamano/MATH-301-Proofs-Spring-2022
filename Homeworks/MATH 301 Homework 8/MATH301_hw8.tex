\documentclass[12pt]{article}
\usepackage[pdftex]{graphicx}
\usepackage{amsmath}
\usepackage[final]{pdfpages}


\usepackage{amssymb}
\pagestyle{empty}
\author{Chris Camano: ccamano@sfsu.edu}
\title{MATH 301  Homework 8 }
\date{3/25/2022}

\topmargin -0.6in
\headsep 0.40in
\oddsidemargin 0.0in
\textheight 9.0in
\textwidth 6.5in

\newcommand{\econst}{\mathrm{e}}
\newcommand{\diff}{\mathrm{d}}
\newcommand{\dwrt}[1]{\frac{\diff}{\diff #1}}
%%%%%%Macros for 425%%%%%%%%%%%%%%%%%%%
\newcommand{\q}{\quad}
\newcommand{\tab}{\\\\}
\renewcommand{\labelenumi}{\alph{enumi})}
\newcommand{\sect}[1]{\section*{#1}}

%%%%%%Vector Spaces%%%%%%%%%%%%%%%%%%%
\newcommand{\R}{\mathbb{R}}
\newcommand{\C}{\mathbb{C}}
\newcommand{\F}{\mathbb{F}}
\newcommand{\Z}{\mathbb{Z}}
\newcommand{\rtwo}{\mathbb{R}^2}
\newcommand{\mxn}{{mxn}}

%%%%%%Sets and common phrases%%%%%%%%%
\newcommand{\Axb}{\textbf{Ax=b} }
\newcommand{\Axz}{\textbf{Ax=0} }
\newcommand{\dim}{\text{dim}}
\newcommand{\lc}{linear combination }
\newcommand{\let}{\text{Let }}
\newcommand{\tf}{\therefore}
%%%%%%%%%%%%%%%%%%%%%%%%%%%%%%%%%%%%%%%
\everymath={\displaystyle}


\begin{document}
\maketitle
\begin{center}
  \textbf{Chapter 7) 4,6,8,12,16,20}\\
\end{center}



\sect{7.4}
\begin{proof}
  +++\\
  Given an integer a, then $a^3+4a+5$ is odd if and only if a is even.\\
  \textbf{If and only if proof}\\
  \textbf{ $a^3+4a+5$ is odd $\rightarrow$ a is even}\\
  \textbf{Contrapositive proof: }
  \textbf{ if a is odd $\rightarrow a^3+4a+5$ is even  }\\
  \begin{align*}
    &\text{Let }a=2k+1, k\in \Z \therefore\\
    &a^3+4a+5=(2k+1)^2+4(2k+1)+5=\\
    &4k^2+4k+1+8k+4+5=\\
    &4k^2+12k+10=\\
    &2(2k^2+6k+5)
  \end{align*}
  Thus it can be shown that when a is an odd number the experession $a^3+4a+5$ can be expressed as 2 times some integer of the form: $2k^2+6k+5, k \in \Z$\\
  \textbf{ a is even $\rightarrow$  $a^3+4a+5$ is odd }
  \begin{align*}
    &\text{Let }a =2k, k \in \Z \therefore\\
    &a^3+4a+5=(2k)^3+4(2k)+5=\\
    &8k^3+8k+5=\\
    &8k^3+8k+4+1=\\
    &2(4k^3+4k+2)+1\\
    &\text{Let }m=4k^3+4k+2\\
    &a^3+4a+5=2m+1
  \end{align*}
  Meaning when a is even $a^3+4a+5$ is odd.\\
  I found this poof to be realy intuitive, this felt like a good first problem that was an extention of the direct proofs for conditionals we were doing in previous chapters
\end{proof}\\
\sect{7.6}
\begin{proof}
  +++\\
  Suppose x,y $\in \R$ Then $x^3+x^2y=y^2+xy$ if and only if $y=x^2$ or y=-x\\
  \textbf{If and only if proof}\\
  \textbf{If $x^3+x^2y=y^2+xy \rightarrow$ $\quad y=x^2$ or $y=-x$ }\\
  \textbf{Direct Proof:}\\
  \begin{align*}
    &x^3+x^2y=y^2+xy\\
    &x^2(x+y)=y(x+y)\\
    &x^2(x+y)-y(x+y)=0\\
    &(x^2-y)(x+y)=0\\
    &\therefore y=-x \lor y=x^2
  \end{align*}
  \textbf{If $y=x^2$ or $y=-x \rightarrow$ $\quad x^3+x^2y=y^2+xy$ }\\
  \textbf{Direct Proof:}\\
  \begin{align*}
    &\text{Let } y=x^2\therefore \\
    &x^3+x^2y= x^3+x^4\\
    &y^2+xy=(x^2)^2+xx^2=x^3+x^4\\
    &\therefore x^3+x^2y=y^2+xy
  \end{align*}
This proof was very fun becuase it got very algebraic compared to some of the other problems in this set. I really enjoyed how for the second direct proof I could prove both sides of the equation once I understood the substitution.
\end{proof}\\

\sect{7.8}
\begin{proof}
  +++\\
  Suppose a,b $\in \Z$ Prove that $a \equiv b(\mod 10)$ if and only if $a \equiv b(\mod 2)$ and $a \equiv b(mod 5)$\\\\
  \textbf{If $a \equiv b(\mod 10)\rightarrow $$a \equiv b(\mod 2)$ and $a \equiv b(\mod 5)$ }\\
  If $a \equiv b(\mod 10)$ then $10|(a-b)$. This is to say that the a - b is some multiple of 10 in math:
  \[
    a-b=10k, k \in \Z
  \]
  \begin{align*}
    &a-b=5(2m),m \in \Z\therefore\\
    & a-b =5(n), n=2m \in \Z \therefore\\
    &a \equiv b \mod 5
  \end{align*}
  \text{Likewise: }
  \begin{align*}
    &a-b=2(5l), l \in \Z\therefore\\
    &a-b=2(p), p=5l \in \Z \therefore\\
    &a \equiv b \mod 2 \\
    &\blacksquare
  \end{align*}
  \\
  \textbf{If $a \equiv b(\mod 2)$ and $a \equiv b(\mod 5)\rightarrow$$a \equiv b(\mod 10)$}
  If $a \equiv b mod 2$ and $a \equiv b mod 5$ then:
  \[
    2|a-b \quad \land \quad 5|a-b
  \]
  If two numbers divide an integer this implies that their product does as well. Therefore $10|a-b, a \equiv b \mod 10$  \blacksquare\\
  This was the first problem I have ever worked on involving the modulo operator and considering the notion of congrunecy. I researched how to frame this problem and it soon became apparent how to solve using the definition of congrunecy.
\end{proof}\\
\sect{7.12}
\begin{proof}
  +++\\
 There exists a positive real number x for which $x^2<\sqrt{x}$\\
 \textbf{Existential proof}:
 Consider the number $\frac{1}{2}$
  \[
    \Big(\frac{1}{2}\Big)^2< \sqrt{\frac{1}{2}}
  \]
  \[
    .25<.707107...
  \]
 \end{proof}\\
 This proof was very straight forward and I was able to provide an example after considering a few numbers and the byproduct of calculating their square root.
\sect{7.16}
\begin{proof}
  +++\\
  Suppose $a,b, \in \Z$. If ab is odd then $a^2+b^2$ is even.\\
  \textbf{Direct Proof}\\
    If ab is odd then a and b are both odd as there does not exist two even numbers with an odd product or an odd an even number that have an odd product.Proof below: \\
    \[
      \textbf{Let }a=2k+1,k\in \Z \quad b=2m+1, m\in \Z
    \]
    \begin{align*}
      &(2k+1)(2m+1)=\\
      &4km+2m+2k+1=\\
      &2(2km+m+k)+1
    \end{align*}
  Therefore the product of two odd numbers is odd.
    \begin{align*}
      &a^2+b^2=(2k+1)^2+(2m+1)^2=\\
      &4k^2+4k+1+4m^2+4m+1=\\
      &2(2k^2+2k+2m^2+2m+1)
    \end{align*}
    $a^2+b^2$ can be expressed as 2 times some integer therefore it is even.
\end{proof}\\
This problem felt slightly out of place for me since it seemed solvable using a direct proof method. Please let me know if I was supposed to solve this differently.
\sect{7.20}
\begin{proof}
  +++\\
  There exists an $n \in \mathbb{N}$ for which $11|(2^n-1)$\\
  \textbf{Existential proof}:

  \begin{figure}[h]
    \caption{Consider the numbers of the form 10k, $k \in \mathbb{W}$(whole numbers )
    these are solutions to this problem, for example 10,20,30 are solutions.
    Answers generated using c++ computation with the following code:}
   \includegraphics[,width=\textwidth]{Capture.png}
  \end{figure}

  \[
    2^{40}-1=1099511627776 \quad \frac{1099511627776}{11}=99955602525
  \]
 This was a fun problem because many of the exestential proofs can be verified computationally. Attached is my code.
\end{proof}\\
\end{document}
