\documentclass[12pt,a4paper]{article}
\usepackage[utf8]{inputenc}
\usepackage{amsmath}

\usepackage{amsfonts}
\usepackage{amssymb}
\usepackage{tikz}
\usepackage{amsmath}
\usepackage{amssymb}
\usepackage{pgfplots}
\usepackage{nccmath}
\usepackage{mathtools}
\usepackage{pgfplots}
\usepackage{mathtools,amssymb}
\usepackage{tikz}
\usepackage{xcolor}
\pgfplotsset{compat = newest}
\author{Chris Camano: ccamano@sfsu.edu}
\title{MATH301 Homework 2 }
\date{2/8/2022}
% Margins
\topmargin=-0.45in
\evensidemargin=0in
\oddsidemargin=0in
\textwidth=6.5in
\textheight=9.0in
\headsep=0.25in
\newcommand{\q}{\quadd}
\renewcommand{\labelenumi}{\alph{enumi})}
\newcommand{\rtwo}{$\mathbb{R}^2$}
\newcommand{\C}{$\mathbb{C}$}
\newcommand{\tf}{\therefore}
\begin{document}
\maketitle
\section{Section 1.5 Problems:4a,4e,4f,4g,8}\\
$$ \text{Suppose} A=\{b,c,d\}\text{ and } B=\{a,b\} \text{ Find}:$$
\textbf{1.5:4a +++}\\
\[
  (A \times B) \cap ( B \times B)
\]
\begin{align*}
  (A \times B)=\{(b,a),(b,b),(c,a),(c,b),(d,a),(d,b)\}\\
  (B \times B)=\{(a,a),(a,b),(b,a),(b,b))\}\\
  \therefore \\
    (A \times B) \cap ( B \times B)=\{(b,a),(b,b)\}
\end{align*}\\\\
For this assignment I feel that I understood the task at hand and writing out the complete cartesian product makes it very easy to identify the intersection of the two sets.\\\\
\textbf{1.5:4e+++}\\
\[
  (A \times B) \cap B
\]
\begin{align*}
  (A \times B)=\{(0,1),(0,2),(1,1),(1,2)\}\\
  1,2 \notin (A \times B)\quadd \therefore (A \times B) \cap B=\emptyset
\end{align*}
This problem is very similar to the previous and I found that becuase there are no ordered pairs within B the intersection should be the empty set.\\\\
\textbf{1.5:4f +++}\\
\[
  \mathcal{P}(A)\cap \mathcal{P}(B)
\]
\begin{align*}
  \mathcal{P}(A)=\{\emptyset, \{b\},\{c\},\{d\},\{b,c\},\{b,d\},\{c,d\},\{b,c,d\}\\
  \mathcal{P}(B)=\{\emptyset,\{a\},\{b\},\{a,b\}\\ \therefore
    \mathcal{P}(A)\cap \mathcal{P}(B)=\{\emptyset,\{b\}
\end{align*}
Like problem 1 I found that a complete expansion of the sets being examined really assists the action of identifying the intersection between the two sets.\\\\
\textbf{1.5:4g +++}\\
\[
    \mathcal{P}(A) - \mathcal{P}(B)
\]
\begin{align*}
  \mathcal{P}(A)=\{\emptyset, \{b\},\{c\},\{d\},\{b,c\},\{b,d\},\{c,d\},\{b,c,d\}\\
  \mathcal{P}(B)=\{\emptyset,\{a\},\{b\},\{a,b\}\\\tf
  \mathcal{P}(A)-\mathcal{P}(B)=\{x|x\in\mathcal{P}(A),x\notin\mathcal{P}(B)\}=\{\{c\},\{d\},\{b,c\},\{b,d\},\{c,d\},\{b,c,d\}\\
\end{align*}
This problem was very fun to type in $\latex$ and I think that the set builder notation expressed at the end of my work is a nice description of the problem. here I am attempting to describe the set of elements in the powerset of A that are not within B.\\\\
\textbf{1.5:8 +++}\\
See attached illustration, figure 1\\
This problem was very graphically intuitive and examining the regions described by the sets led to clear decision making for which sections I felt needed to be shaded.\\\\

\section{Section 1.7 Problems 6,8,12,14}\\
\textbf{1.7.6+++}\\
See attached illustration, figure 2\\
\textbf{1.7.8+++}\\
See attached illustration, figure 3\\
For problems 1.7.6 and 1.7.8 I found that focusing on building the section procedurally really helped my work. By methodically shading the regions described by the set algebra I found that both of these examples were equivilant.\\\\
\textbf{1.7.12+++}\\
The expression that describes this set it:
\[
(A-B)\cup (B\cap C)
\]
For this problem I found that it was very useful to first about how I would shade just the set of A. doing so made the intersection with the intersection of B and C very easy and then it was evident that I needed to subtract B from the set A.\\\\
\textbf{1.7.14+++}\\
The expression that describes this set is :
\[
  (A\cap B\cap C)\cap \Big((A-C)\cap(A-B)\Big)
\]
This problem required quite alot of thinking for me compared to the others. I found that the best way to conceptualize this problem was to start with the full intersection of the three sets and then find a concise description of just the are within set A to union with the intersection.\\\\
\section{4 dimensional cube illustration+++}\\
See attached illustration, figure 4.\\
I really enjoyed watching this video and feel that the powerset does seem to naturally want to exist as a cube or higherdimensional object with objective structure.\\\\
\section{Bonus ++++}
\end{document}
