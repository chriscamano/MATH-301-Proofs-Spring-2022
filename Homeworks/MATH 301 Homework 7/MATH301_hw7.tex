\documentclass[12pt]{article}
\usepackage[pdftex]{graphicx}
\usepackage{amsmath}
\usepackage{amssymb}
\pagestyle{empty}
\author{Chris Camano: ccamano@sfsu.edu}
\title{MATH 301 HW 7}
\date{3/8/2022}
\topmargin -0.6in
\headsep 0.40in
\oddsidemargin 0.0in
\textheight 9.0in
\textwidth 6.5in

\newcommand{\econst}{\mathrm{e}}
\newcommand{\diff}{\mathrm{d}}
\newcommand{\dwrt}[1]{\frac{\diff}{\diff #1}}
%%%%%%Macros for 425%%%%%%%%%
\newcommand{\q}{\quad}
\newcommand{\tab}{\\\\}
\renewcommand{\labelenumi}{\alph{enumi})}
\newcommand{\sect}[1]{\section*{#1}}
\newcommand{\bb}[1]{\mathbb*{#1}}
\newcommand{\cal}[1]{\mathcal*{#1}}
\newcommand{\R}{\mathbb{R}}
\newcommand{\C}{\mathbb{C}}
\newcommand{\F}{\mathbb{F}}
\newcommand{\Z}{\mathbb{Z}}
\newcommand{\rtwo}{\mathbb{R}^2}
\newcommand{\mxn}{{mxn}}

\newcommand{\Axb}{\textbf{Ax=b} }
\newcommand{\Axz}{\textbf{Ax=0} }
\newcommand{\dim}{\text{dim}}
\newcommand{\lc}{linear combination }
\newcommand{\tf}{\therefore}
%%%%%%%%%%%%%%%%%%%%%%%%%%%%%
\begin{document}
\maketitle
\sect{Problem list:}
\begin{center}
   Chapter 6, Problem 4\\
   Chapter 6, Problem 6\\
   Chapter 6, Problem 10\\
   Chapter 6, Problem 16\\
   Let n be an integer.
  If $n^2$ is a multiple of 3 prove that n is a multiple of 3.
\end{center}


\sect{Problem 6.4}\\
+++
\begin{proof}
  Prove that $\sqrt{6}$ is irrational\\
  Suppose $\sqrt{6}$ is rational,\\
  \begin{align*}
    & \text{This is to say that } \sqrt{6}\text{ can be expressed as }=\frac{a}{b},a,b \in \Z,a\neq b,b\neq 0 \\
    & \text{Squaring both sides: }6=\frac{a^2}{b^2}\\
    & 6b^2=a^2\\\\
    & 2(3b^2)=a^2\quad \therefore a^2 \text{ is even}\\
    &\text{ Since }a^2 \text{ is even }a \text{ is even}^{[1]}\\
    &\text{If }a=2c \text{ then}\\
    & 6b^2=(2c)^2\\
    & 6b^2 =4c^2\\
    & 3b^2=2c^2\\
  \end{align*}
  Meaning that $3b^2$ is even as it can be expressed as two times an interger which implies b is even } $^{[1]}$ \\
  Since a is even and b is even this suggests that they share a common divisor of 2 which contradicts our original assumption that $\sqrt{6}$ can be expressed as a ratio of two numbers with no common divisor\\\\
  This problem was very intitive for me and I like the trend of showing that a number cannot have both parities, it seems like a very consistent and safe methodology for providing contradiction.
\end{proof}\\
\sect{Problem 6.6}\\
+++
\begin{proof}
  If a,b$\in \Z$, then $a^2-4b-2 \neq 0$\\
  Suppose if a,b$\in \Z$, then $a^2-4b-2 =0$,\\
  \begin{align*}
    &a^2=4b+2\\
    &a^2=2(2b+1)\quad \therefore a^2 \text{ is even as it can be written as 2 times some integer}\\
    &\text{Let } a= 2m , m \in \Z \\
    & (2m)^2=4b+2\\
    &4m^2-4b-2=0\\
    &2m^2-2b-1=0\\
    &2(m^2-b)=1\\
    &(m^2-b)=\frac{1}{2}\\
    &b=m^2-\frac{1}{2}\\
  \end{align*}
  Here we have shown that b must be some integer m squared minus one half. This contradicts our original statement that b is an integer as there does not exist an integer m such that subtracting one half from its square returns an integer.\\\\
  This problem used another of my favorite ways of showing contradiction, which is that through the assumption of the negated statement we are led to some description of a member of a set that defies the axioms of the selected field. Here we see that for this to be true the intergers would contain rational numbers
\end{proof}\\
\sect{Problem 6.10}\\
+++
\begin{proof}
  There exist no integers a and b for which $21a+30b=1$\\
  Suppose there exist integers a and b for which $21a+30b=1$\\
  Note that these two terms share a coefficient with a common factor of three. Through algebraic manipulatio it can be shown that dividing both sides of the equation reveals that we are suggesting that two integers would sum to rational number which does not obey the rules of the integers which are closed under addition.
  \[
   7a+10b=\frac{1}{3}
  \]

  \\\\ For this problem i chose to omit most of the algebra as the result felt better described with english. I hope this choice is okay. My understanding is that the negation can be disproven simply by dividing by three and showing that we are suggesting two integers can sum to a rational number.
\end{proof}\\
\sect{Problem 6.16}\\
+++
\begin{proof}
  If a and b are positive real numbers, then a+b $\geq 2\sqrt{ab}$\\
  Suppose a and b are positive real numbers, and a+b $< 2\sqrt{ab}$

  \begin{align*}
    & a+b < 2\sqrt{ab}\\
    & (a+b)^2 < 4ab\\
    &a^2+2ab+b^2<2ab\\
    &a^2-2ab+b^2<0\\
    &(a-b)^2<0\\
    &a-b<0\\
    &\nexists a,b \in \Z : (a-b)^2<0, \text{ as }\forall a,b \in \Z \quad a-b \geq 0
  \end{align*}
  This is to say that there do not exist integers a and b such that a -b is less than zero. The desired result follows.\\\\
  I enjoyed this problem the most out of this problem set and I found this contradiction to be very unique and interesting to think about.
\end{proof}\\
\sect{If $n^2$ is a multiple of 3 prove that n is a multiple of 3.}\\
\begin{proof}
  Suppose $n^2$ is a multiple of 3 and n is not a multiple of 3,\\
  \begin{align*}
    &\text{Let }n^2=3k, k \in \Z\\
  \end{align*}
  If n is not a multiple of three this implies that n exists as some number between a multiple of three and the next multiple. This range consists of two numbers meaning that n can only be of the forms:
  \[
    n=3m+1 \lor n=3m+2\quad
  \]
  If this is the case we are suggesting that $n^2$ is one of the following:
  \begin{align*}
    n^2= (3m+1)^2=9m^2+6m+1=3(3m^2+2m)+1 \\
    n^2=(3m+2)^2=9m^2+12m+4=3(3m^2+4m+1)+1\\
  \end{align*}
  In both cases it is evident that $n^2$ is not a multiple of three which contradicts our starting statment. The desired result follows.\\\\
  Out of my proofs on this assignment this one felt the weakest. While the statements supplied are all true and the logic feels sufficient to prove a contradiction, I felt that my general approcah was a little "loose" and could have been benifited by more rigor.
\end{proof}\\


\sect{[1]}
\begin{proof}
  Suppose $n\in \Z$. If $n^2$ is even then n is even\\
  \textbf{Proof:}\\
  Suppose n is odd. Then:
  \[
    n=2k+1, k\in \Z
  \]
  We now wish to prove that if n is odd $n^2$ is odd.
  \begin{align*}
      &(2k+1)^2=4k^2+4k+1\\
      &\qquad\qquad = 2(2k^2+2k)+1\\
      &\qquad\qquad k \in \Z \therefore (2k^2+2k)\in \Z\\
      &\qquad\qquad\text{let }2k^2+2k=m\therefore\\
      &\qquad\qquad(2k+1)^2=2(m)+1
  \end{align*}
Meaning that when n is odd $n^2$ is odd. Through logical equivilancy this implies that when $n^2$ is even n is even.
\\

\end{proof}\tab\tab
\end{document}
