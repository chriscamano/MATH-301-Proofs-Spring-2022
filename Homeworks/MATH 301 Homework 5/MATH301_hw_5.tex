\documentclass[12pt]{article}
\usepackage[pdftex]{graphicx}
\usepackage{amsmath}
\usepackage{amssymb}
\pagestyle{empty}
\author{Chris Camano: ccamano@sfsu.edu}
\title{MATH 301 Homework 5 }
\date{2/24/2022}

\topmargin -0.6in
\headsep 0.40in
\oddsidemargin 0.0in
\textheight 9.0in
\textwidth 6.5in

\newcommand{\econst}{\mathrm{e}}
\newcommand{\diff}{\mathrm{d}}
\newcommand{\dwrt}[1]{\frac{\diff}{\diff #1}}
%%%%%%Macros for 425%%%%%%%%%
\newcommand{\q}{\quad}
\newcommand{\tab}{\\\\}
\renewcommand{\labelenumi}{\alph{enumi})}
\newcommand{\sect}[1]{\section*{#1}}

\newcommand{\R}{\mathbb{R}}
\newcommand{\C}{\mathbb{C}}
\newcommand{\Z}{\mathbb{Z}}
\newcommand{\F}{\mathbb{F}}
\newcommand{\rtwo}{\mathbb{R}^2}
\newcommand{\mxn}{{mxn}}

\newcommand{\Axb}{\textbf{Ax=b} }
\newcommand{\Axz}{\textbf{Ax=0} }
\newcommand{\dim}{\text{dim}}
\newcommand{\lc}{linear combination }
\newcommand{\let}{\text{Let }}
%%%%%%%%%%%%%%%%%%%%%%%%%%%%%
\everymath={\displaystyle}


\begin{document}
\maketitle
\sect{Problem 4-4}+++
\begin{proof}
  Use the method of direct proof to prove the following statment:\tab
  Suppose x,y $\in \Z$. If x and y are odd then xy is odd.\tab
  \textbf{Proof:}\\
  Suppose x,y are odd therefore:
  \[
    x=(2k+1),y=(2m+1)
  \]
  $$k,m\in \Z$$
  \begin{align*}
      &x(y)=(2k+1)(2m+1)\tab
      &x(y)=4km+2k+2m+1\tab
      &x(y)=2(2km+k+m)+1\tab
      &k,m\in \Z \therefore (2km+k+m) \in \Z\tab
      &\text{Let }(2km+k+m)=n,n\in \Z \therefore x(y)=2n+1\\
      &\therefore x(y) \text{is odd}\\
  \end{align*}}
  This problem investigated the effect of multiplying two odd numbers together. I feel really confident with my work here, and fully understand the problem.
\end{proof}\tab

\sect{Problem 4-6}+++
\begin{proof}
  Use the method of direct proof to prove the following statment:\tab
  Suppose a,b,c $\in \Z$. If $a|b$ and $a|c$ then $a|(b+c)$\tab
  \textbf{Proof:}\\
  \begin{align*}
    \text{Suppose } a|b \text{ and } a|c \therefore\\
     b=a(m),m\in Z \quad c=a(n),n\in \Z\\
  \end{align*}
  If $a|(b+c)$ then we must show that $(b+c)= a(k), k \in \Z$\\
  If $ b=a(m),m\in Z \quad$and $ c=a(n),n\in \Z$ then
  \begin{align*}
    (b+c)=a(m)+a(n)=a(m+n)\\
    \text{Let }m+n= k \therefore\\
     (b+c)=a(k)
  \end{align*}
  Since b+c can be expressed as a multiple of a when $a|b$ and $a|c$ ,it is proved that $a|(b+c)$.\\
  This problem was my first exploring the concepts of division. I feel that I took a few notational liberties with my work, but that my take on this proof was largely justified. The bulk of this work went into demonstrating that b+c can be expressed as a multiple of a.
\end{proof}\tab

\sect{Problem 4-14}+++
\begin{proof}
  Use the method of direct proof to prove the following statment:\tab
  If n$\in \Z$ then $5n^2+3n+7$ is odd (try cases)\tab
  For this problem we must test both parities of n: odd and even.\\
  \textbf{Proof: n is odd}
  \[
    \text{Suppose }n=2k+1, k \in \Z
  \]
  Then the statement $5n^2+3n+7$ can be expressed as:
  \[
    5(2k+1)^2+3(2k+1))+7
  \]
  \begin{align*}
      &5(2k+1)^2+3(2k+1))+7= 5(4k^2+4k+1)+6k+10\\
      &\qquad\qquad\qquad\qquad\qquad\qquad =20k^2+26k+10+1\\
      &\qquad\qquad\qquad\qquad\qquad\qquad =2(10k^2+13k+5)+1\\
      &\qquad\qquad\qquad\qquad\qquad\qquad \qquad k \in \Z \therefore (10k^2+13k+5)\in \Z\\
      &\qquad\qquad\qquad\qquad\qquad\qquad\qquad    \text{let } (10k^2+13k+5)=m\quad \therefore\\
      &\qquad\qquad\qquad\qquad\qquad\qquad =\textbf{2(m)+1}\therefore\\
      &\textbf{When n is odd $5(n)^2+3(n)+7$ is odd }
  \end{align*}
  \textbf{Proof: n is even}
  \[
    \text{ Suppose }n=2k, k \in \Z
  \]
  Then the statement $5n^2+3n+7$ can be expressed as:
  \[
      5(2k)^2+3(2k)+7
  \]
  \begin{align*}
    &5(2k)^2+3(2k)+7=5(4k^2)+6k+7\\
    &\qquad\qquad\qquad\qquad=10k^2+6k+6+1\\
    &\qquad\qquad\qquad\qquad=2(5k^2+3k+3)+1\\
    &\qquad\qquad\qquad\qquad k\in \Z \therfore (5k^2+3k+3) \in \Z\\
    &\qquad\qquad\qquad\qquad\text{let }(5k^2+3k+3)=m\\
    &\qquad\qquad\qquad\qquad=2(m)+1\therefore\\
    &\textbf{When n is even $5(n)^2+3(n)+7$ is odd }
  \end{align*}
  Therfore regardless of the parity of n $(5(n)^2+3(n)+7)$ is odd
  \\
  This question was difficult for me becuase it pushed me to consider all cases presented for the problem. Luckily the structure of the proof was very symmetric and I am confident in my work.
\end{proof}\tab

\sect{Problem 5-2}+++
\begin{proof}
  Prove the following statments with contrpositive proof.(In each case, think about how a direct proof would work. In most cases contrpositive is easier)\tab
  Suppose $n\in \Z$. If $n^2$ is even then n is even\\
  \textbf{Proof:}\\
  Suppose n is odd. Then:
  \[
    n=2k+1, k\in \Z
  \]
  We now wish to prove that if n is odd $n^2$ is odd.
  \begin{align*}
      &(2k+1)^2=4k^2+4k+1\\
      &\qquad\qquad = 2(2k^2+2k)+1\\
      &\qquad\qquad k \in \Z \therefore (2k^2+2k)\in \Z\\
      &\qquad\qquad\text{let }2k^2+2k=m\therefore\\
      &\qquad\qquad(2k+1)^2=2(m)+1
  \end{align*}
Meaning that when n is odd $n^2$ is odd. Through logical equivilancy this implies that when $n^2$ is even n is even.
\\
This problem was a good introduction to the contrapositive proof method I am very confident with my result. It seems to me that the problem was made to investigate this new technique!
\end{proof}\tab\tab

\sect{Problem 5-16}+++
\begin{proof}
  Prove the following statements using either direct or contrapositive proof. \tab
  Suppose x,y$\in \Z$ If x+y is even, then x and y have the same parity.\\
  Suppose that  x and y do not have the same parity. This is to say that:
  \[
    x=2k \quad y=2m+1 \quad k,m\in \Z
  \]
  or
  \[
    x=2m+1 \quad y=2k \quad k,m\in \Z
  \]
  \textbf{Proof: x is even y is odd}:
  Suppose $x=2k \quad y=2m+1 \quad k,m\in \Z$:
  \begin{align*}
    &x+y=2k+2m+1\\
    &x+y=2(k+m)+1\\
    &k,m \in \Z \therefore k+m \in \Z\\
    & \text{Let } n=k+m \therfore:\\
    &x+y=2(n)+1
  \end{align*}
  Consequently x+y is odd . \\
  Therfore x+y is not even.\\
  \textbf{Proof: y is even x is odd}:
  Suppose $y=2k \quad x=2m+1 \quad k,m\in \Z$:
  \begin{align*}
    &y+x=2k+2m+1\\
    &y+x=2(k+m)+1\\
    &k,m \in \Z \therefore k+m \in \Z\\
    & \text{Let } n=k+m \therfore:\\
    &y+x=2(n)+1
  \end{align*}
  Consequently x+y is odd . \\
  Therfore x+y is not even.\\
  Since the negation of the original statment created two sub problems this problem had to be solved through cases and reminded of two previously solved in this assignment. I am confident in my answer.
\end{proof}\tab\tab
\end{document}
