\documentclass[12pt]{article}
\usepackage[pdftex]{graphicx}
\usepackage{amsmath}
\usepackage{amssymb}
\pagestyle{empty}
\author{Chris Camano: ccamano@sfsu.edu}
\title{MATH 301  Homework 11 }
\date{4/25/2022}

\topmargin -0.6in
\headsep 0.40in
\oddsidemargin 0.0in
\textheight 9.0in
\textwidth 6.5in

\newcommand{\econst}{\mathrm{e}}
\newcommand{\diff}{\mathrm{d}}
\newcommand{\dwrt}[1]{\frac{\diff}{\diff #1}}
%%%%%%Macros for 425%%%%%%%%%%%%%%%%%%%
\newcommand{\q}{\quad}
\newcommand{\tab}{\\\\}
\renewcommand{\labelenumi}{\alph{enumi})}
\newcommand{\sect}[1]{\section*{#1}}

%%%%%%Vector Spaces%%%%%%%%%%%%%%%%%%%
\newcommand{\R}{\mathbb{R}}
\newcommand{\C}{\mathbb{C}}
\newcommand{\F}{\mathbb{F}}
\newcommand{\Z}{\mathbb{Z}}
\newcommand{\rtwo}{\mathbb{R}^2}
\newcommand{\mxn}{{mxn}}

%%%%%%Sets and common phrases%%%%%%%%%
\newcommand{\Axb}{\textbf{Ax=b} }
\newcommand{\Axz}{\textbf{Ax=0} }
\newcommand{\dim}{\text{dim}}
\newcommand{\lc}{linear combination }
\newcommand{\let}{\text{Let }}
\newcommand{\tf}{\therefore}
%%%%%%%%%%%%%%%%%%%%%%%%%%%%%%%%%%%%%%%
\everymath={\displaystyle}


\begin{document}
\maketitle
\begin{center}
  Chapter 10: 2,10,22,30,32,20,42,
\end{center}
\sect{Problem 10.2}
  \begin{proof}
  Prove that 1+2+3+4...+n=$\frac{n^2+n}{2}$ for all $n \in \Z^+$\\
  \textbf{Proof by induction}:\\
  \textbf{Base case:}n=3
  $$1+2+3=\frac{9+3}{2}$$\\
  $$6=6$$\\
  \textbf{Inductive Hypothesis:}
  \[
    P(k)=\sum_{n=1}^kn=\frac{k^2+k}{2}
  \]
  \textbf{Proving P(k+1)}
  \begin{align*}
    &P(k+1)=\sum_{n=1}^{k+1}n=\frac{(k+1)^2+k+1)}{2}\\
    &=\sum_{n=1}^{k}n+(k+1)=\frac{(k^2+3k+2)}{2}\\
    &=\frac{k^2+k}{2}+(k+1)=\frac{(k^2+3k+2)}{2}\\
    &\frac{(k^2+3k+2)}{2}=\frac{(k^2+3k+2)}{2}\\
  \end{align*}
  It follows that the hypothesis holds for all k in the positive integers
\end{proof}
\sect{Problem 10.10}
\begin{proof}
  Prove that $3|(5^{2n}-1)\forall n \geq 0, n\in \Z$\\
  \textbf{Proof by induction:}\\
  \textbf{Base Case:n=2}\\
  \[
    5^(4)-1=624=208(3)
  \]
  \textbf{Inductive Hypothesis}\\
  \[
    P(k)=3|5^{2n}-1\therefore 5^{2n}-1=3(x),x\in \Z
  \]
  \textbf{Proving P(k+1)}\\
  \begin{align*}
    &P(k+1)=5^{2(k+1)}-1\\
    &=(5^{2k+2})-1\\
    &=(5^{2k}25)-1\\
    &=25(5^2k)-1\\
    &=25(3x+1)-1\\
    &=75x+24\\
    &=3(25x+8)
  \end{align*}
  Therefore it is shown that 3 divides P(k+1) which implies 3 divides P(k) for all k
\end{proof}
\sect{Problem 10.20}
\begin{proof}
Prove that $(1+2+3+..+n)^2=1^3+2^3+3^3...+n^3$\\
Prove that
\[
  \Big(\sum_{i=1}^ni\Big)^2=\sum_{i=1}^ni^3
\]
Recall from problem 10.2 that we proved \[
  \sum_{i=1}^ni=\frac{n^2+n}{2}
\]
So we now have to consider:
\[
  \Big(\frac{n^2+n}{2}\Big)^2=\sum_{i=1}^ni^3
\]
\[
  \frac{(n^2+n)^2}{4}=\sum_{i=1}^ni^3
\]
\[
  \frac{(n^4+2n^3+n^2}{4}=\sum_{i=1}^ni^3
\]
\textbf{Poof by induction:}\\
\textbf{Base case:}\\
let n =3:
\[
  \frac{(3^4+2(3^3)+3^2}{4}=\sum_{i=1}^3i^3
\]
\[
  \frac{144}{4}=1+8+27
\]
\[
  36=36
\]
\textbf{Inductive hypothesis:}\\
\[
  P(k)=\sum_{i=1}^ki^3=\frac{(k^4+2k^3+k^2}{4}
\]
\textbf{Proving P(k+1)}\\
\begin{align*}
  &P(k+1)=\sum_{i=1}^{k+1}i^3=\frac{((k+1)^4+2(k+1)^3+(k+1)^2}{4}\\
  &P(k+1)=\sum_{i=1}^ki^3+(k+1)^3=\frac{(k^4+4k^3+6k^2+4k+1)+2(k^3+3k^2+3k+1)+(k^2+2k+1)}{4}\\
  &P(k+1)=\sum_{i=1}^ki^3+(k+1)^3=\frac{k^4+6k^3+13k^2+12k+4}{4}\\
  &P(k+1)=\sum_{i=1}^ki^3+(k+1)^3=\frac{k^4+6k^3+13k^2+12k+4}{4}\\
  &P(k+1)=\frac{(k^4+2k^3+k^2}{4}+(k+1)^3=\frac{k^4+6k^3+13k^2+12k+4}{4}\\
  &P(k+1)=\frac{(k^4+2k^3+k^2+4((k+1)^3)}{4}=\frac{k^4+6k^3+13k^2+12k+4}{4}\\
  &P(k+1)=\frac{(k^4+2k^3+k^2+4(k^3+3k^2+3k+1)}{4}=\frac{k^4+6k^3+13k^2+12k+4}{4}\\
  &P(k+1)=\frac{k^4+6k^3+13k^2+12k+4}{4}=\frac{k^4+6k^3+13k^2+12k+4}{4}
\end{align*}
Thus it follows by mathematical induction that P(k) is true for all natural numbers n.
\end{proof}
\sect{Problem 10.22}
\begin{proof}
If $n\in \mathbb{N}$ then :
\[
  \prod_{i=1}^n\Big[1-\frac{1}{2^i}\Big]\geq\frac{1}{4}+\frac{1}{2^{n+1}}
\]
\textbf{Proof by induction:}\\
\textbf{Base case:}\\
Let n =1:
\[
  \prod_{i=1}^2\Big[1-\frac{1}{2^i}\Big]\geq\frac{1}{4}+\frac{1}{2^{2+1}}
\]
\[
  (1-\frac{1}{2})(1-\frac{1}{2^2})\geq \frac{1}{4}+\frac{1}{2^3}
\]\[
  \frac{3}{8} \geq\frac{3}{8}
\]
\textbf{Inductive Hypothesis}\\
\[
  P(k)=  \prod_{i=1}^k\Big[1-\frac{1}{2^i}\Big]\geq\frac{1}{4}+\frac{1}{2^{k+1}}
\]
\textbf{Proving P(k+1)}\\
\begin{align*}
  &P(k+1)=  \prod_{i=1}^{k+1}\Big[1-\frac{1}{2^i}\Big]\geq\frac{1}{4}+\frac{1}{2^{k+2}}
  \\
  &P(k+1)=  \prod_{i=1}^{k}\Big[1-\frac{1}{2^i}\Big](1-\frac{1}{2^{k+1}})\geq\frac{1}{4}+\frac{1}{2^{k+2}}
  \\
  &\Big(\frac{1}{4}+\frac{1}{2^{k+1}}\Big)(1-\frac{1}{2^{k+1}})\geq\frac{1}{4}+\frac{1}{2^{k+2}}
  \\
  &\frac{1}{4}-\frac{1}{4(2^{k+1})}+\frac{1}{2^{k+1}}-\frac{1}{2^{k+1}(2^{k+1})}\geq\frac{1}{4}+\frac{1}{2^{k+2}}
  \\
  &\frac{1}{4}-\frac{1}{2^{k+1}}\Big(-\frac{1}{4)}+1-\frac{1}{(2^{k+1})}\Big)\geq\frac{1}{4}+\frac{1}{2^{k+2}}
  \\
  &\frac{1}{4}-\frac{1}{2^{k+1}}\Big(\frac{3}{4}-\frac{1}{(2^{k+1})}\Big)\geq\frac{1}{4}+\frac{1}{2^{k+2}}
  \\
  \end{align*}


  \text{We are proving for values of k past the base case so we know that k must be greater than 1 }
  \\
  $\text{This implies k}>1 \therefore \frac{1}{2^{k+1}}<\frac{1}{2^2}$
  \\
  \text{Since increasing the exponent of the 2 in the denominator would  }
  \\
  \text{lower the value of the rational number.switching the sign and concequently the inequality:}

  \begin{align*}
  &\frac{1}{2^{k+1}}<\frac{1}{2^2}=-\frac{1}{2^{k+1}}>-\frac{1}{4}
  \\
  &\frac{1}{4}-\frac{1}{2^{k+1}}\Big(\frac{3}{4}-\frac{1}{4}\Big)\geq\frac{1}{4}+\frac{1}{2^{k+2}}
  \\
  &\frac{1}{4}-\frac{1}{2^{k+1}}\Big(\frac{1}{2}\Big)\geq\frac{1}{4}+\frac{1}{2^{k+2}}
  \\
  &\frac{1}{4}-\frac{1}{2^{k+2}}\geq\frac{1}{4}+\frac{1}{2^{k+2}}\\
\end{align*}
Since
\end{proof}
\sect{Problem 10.30}
\begin{proof}

\end{proof}
\sect{Problem 10.32}
\begin{proof}

\end{proof}
\sect{Problem 10.42}
\begin{proof}

\end{proof}
\end{document}
