\documentclass[12pt,a4paper]{article}
\usepackage[utf8]{inputenc}
\usepackage{amsmath}

\usepackage{amsfonts}
\usepackage{amssymb}
\usepackage{tikz}
\usepackage{amsmath}
\usepackage{amssymb}
\usepackage{pgfplots}
\usepackage{nccmath}
\usepackage{mathtools}
\usepackage{pgfplots}
\usepackage{mathtools,amssymb}
\usepackage{tikz}
\usepackage{xcolor}
\pgfplotsset{compat = newest}
\author{Chris Camano: ccamano@sfsu.edu}
\title{MATH301 Lecture 3 }
\date{2/15/2022}
% Margins
\topmargin=-0.45in
\evensidemargin=0in
\oddsidemargin=0in
\textwidth=6.5in
\textheight=9.0in
\headsep=0.25in
\newcommand{\q}{\quad}
\renewcommand{\labelenumi}{\alph{enumi})}
\newcommand{\R}{\mathbb{R}}
\newcommand{\rtwo}{\mathbb{R}^2}
\newcommand{\C}{\mathbb{C}}
\begin{document}

\maketitle
\section{disucssion of logical operators}

\subsection{negation}\\
The symbol ~ or $\neg$ is used to symbolize the concept of negation, that is to say, "not"
\[
  P: \quad a\wedge b \in \mathbb{Z}
\]
negated form
\[
  \neg P:=\neg(a \wedge b )= \neg a \vee \neg b
\]
\\
\subsection{ $\forall $}\\
$\forall$ is used to refer to all elements in a set. It is a description used to refer to the entirety of a classification of mathematical objects.\\

\[
  \forall n \in \mathbb{Z}
\]
\[
  \neg ( \forall x \quad P(X))=\exists x  \neg P(X)
\]
\[
  \neg ( \exists x \quad P(X))=\forall x  \neg P(X)
\]
\subsection{$\exists$}\\
This is the existence quentifer which states that an object or relationship exists.
\end{document}
