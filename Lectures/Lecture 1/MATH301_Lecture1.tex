\documentclass[12pt,a4paper]{article}
\usepackage[utf8]{inputenc}
\usepackage{amsmath}
\usepackage{amsfonts}
\usepackage{amssymb}
\usepackage{tikz}
\usepackage{amsmath}
\usepackage{amssymb}
\usepackage{nccmath}
\usepackage{mathtools}
\usepackage{pgfplots}
\usepackage{mathtools,amssymb}
\usepackage{tikz}
\usepackage{xcolor}
\pgfplotsset{compat=1.7}
\author{Chris Camano: ccamano@sfsu.edu}
\title{MATH301 Lecture 1}
\date{1/27/2022}
% Margins
\topmargin=-0.45in
\evensidemargin=0in
\oddsidemargin=0in
\textwidth=6.5in
\textheight=9.0in
\headsep=0.25in


\begin{document}
\maketitle
Textbook chapters 1.1-1.5
\section{Sets}
A set is collection of objects. A set is like a bag of certain objects. The objects within a set are referred to as elements.
\\
\subsection{Notation}\\
To refer to a set we generally use capital letters.\\
Given a set A and an alement a we use the notation $a \in A$\\
To describe the contents of a set we use braces as such
\[
  A=\{1,2,3,4,5\}
\]
To talk about a subset we use the following notation:
\[
  \{1,2,3\}\subseteq A
\]
Order of elements does not matter when describing a set. Repitition of elements does not matter when describing a set.
\\
Some important sets within mathematics:
\[
  \mathbb{N}=\{1,2,3,4,...\} \text{ Natural Numbers}\\\\
  \]
  \[
  \mathbb{Z}=\{...,-2,-1,0,1,2,...\} \textbf{ The integers}\\
\]
\[
  \mathbb{E}=\{...,-4,-2,0,2,4,...\} \textbf{The even numbers}
\]
\[
  \emptyset=\{\} \textbf{The empty set}
\]
\subsection{Set builder Notation}
\[
  \mathbb{E}=\{2k| k \in \mathbb{Z}\}
\]
The format for set builder notation is the following
\[
  \textbf{Set}=\{\textbf{formula}| \textbf{Criterion}\}
\]
\\
The Cardinality of a set is the number of elements in that set

\[
  \textbf{Let S be a set } S=\{a,b,c\}, |S|=3
\]
\end{document}
